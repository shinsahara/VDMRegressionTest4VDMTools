独立行政法人情報処理推進機構 (IPA) ソフトウェア・エンジニアリン
グ・センター(SEC) において、ここ数年間にわたって高品質高信頼システムの開発技術に関
して複数の作業部会 (WG) を通して活動を行ってきました。これらの WG の名
称や委員の構成は、状況に応じて変遷してきましたけれども、いわゆるフォー
マルメソッドに基づくソフトウェア開発に関する調査研究ならびに人材育成に
関する活動は一貫して継続してきました。

2010年度以降、フォーマルメソッドに関する入門セミナーないしフォーマルメ
ソッド導入に関するガイダンスセミナーを国内の広島、札幌、熊本、尼崎、名
古屋、東京、盛岡などで開催してきました。この間、セミナー実施経験や受講
者からのアンケート結果などに基づいてセミナー教材の大幅な改訂を行い、
IPA/SEC のホームページ上で「厳密な仕様記述を志すための形式手法入門」と
して公開しています。2012年度からは、改訂版の教材を使用してセミナーを実
施しておりますが、改定前の教材でセミナーを行った都市のなかには改訂版の
教材を用いて再度セミナーを開催したところもあります。

このセミナーは、エンジニア向けと管理者向けに構成を変えて、また、半日コー
ス、一日コース、二日間コースと所要時間も柔軟に組替えられるように工夫し
ました。しかしながら、このセミナーでは、受講者の方々にフォーマルメソッ
ドの概要を伝えることはできても、フォーマルメソッド適用に関する具体的な
手法を理解し習得して頂くには、あまりにも時間が短すぎます。このセミナー
でフォーマルメソッドを代表する一つの手法として取り上げている VDM
(Vienna Development Method) に基づく厳密なシステムの記述に関して、もっ
と具体的に知りたいという受講者の要望も数多く頂きました。

そこで、セミナーで VDM によるシステムの記述の部分を担当している佐原伸氏
に、セミナーで紹介する具体例に関する詳細な副読本としてこの本を執筆して
頂きました。

セミナー教材は、上述のように IPA/SEC のホームページ上で公開されておりま
すので、セミナー受講者でなくても、その教材とこの副読本によってフォーマ
ルメソッド、特に、VDM による厳密なシステム記述に関する基本的な知識と、
システムの厳密な記述の手法に関する理解をえることができます。実のところ、
この副読本は、佐原氏のソフトウェア開発に関する豊富な経験ならびに高い見
識に基づいて、フォーマルメソッドの意義や有用性について実践的な観点から
の説得力ある記述と共に、具体事例を対象として詳細な VDM記述を掲載してお
りますので、フォーマルメソッドの入門書として単独で読んで頂いても十分に
役立つ貴重な書物となっています。

また、IPA/SEC では、フォーマルメソッドのセミナー資料の他にも、「厳密な
仕様記述における形式手法成功事例調査報告書」を公開しています。さらに、
これの報告書に関連して、「厳密な仕様記述入門」という本も作成しました。
これらは、本書の姉妹編とも言うべき有益な資料です。本書と併せてこれらも
ご覧頂くことによって、フォーマルメソッドに基づく高品質高信頼ソフトウェ
アの効率的な開発法に関する理解が深まるとともに、自分たちのところでも実
際にフォーマルメソッドを導入してみようかという気持ちになって頂ければ幸
いです。

 \flushright{荒木 啓二郎}
 \flushright{上流品質技術部会 人材育成WG 主査}
\flushleft{ }