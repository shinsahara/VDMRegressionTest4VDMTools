\documentclass[a4paper,10pt]{jsarticle}
%\documentclass[a4paper,10pt]{jsbook}

\usepackage[dvipdfmx]{graphicx, color}
%\usepackage{folha}
\graphicspath{{image/}}

\usepackage{color}
\usepackage{array}
\usepackage{longtable}
\usepackage{alltt}
\usepackage{graphics}
\usepackage{vpp-nms}
%\usepackage{vpp}
\usepackage{makeidx}
\makeindex

\usepackage{colortbl}

\usepackage[dvipdfmx,bookmarks=true,bookmarksnumbered=true,colorlinks,plainpages=true]{hyperref}

%\AtBeginDvi{\special{pdf:tounicode 90ms-RKSJ-UCS2}}
\AtBeginDvi{\special{pdf:tounicode EUC-UCS2}}

\definecolor{covered}{rgb}{0,0,0}      %black
\definecolor{not-covered}{rgb}{1,0,0}  %red

\setcounter{secnumdepth}{6}
\makeatletter
\renewcommand{\paragraph}{\@startsection{paragraph}{4}{\z@}%
  {1.5\Cvs \@plus.5\Cdp \@minus.2\Cdp}%
  {.5\Cvs \@plus.3\Cdp}%
  {\reset@font\normalsize\bfseries}}
\makeatother

\renewcommand{\sf}{\sffamily \color{blue}}

\newcommand{\syou}{\texttt{<}}
\newcommand{\dai}{\texttt{>}}

%\include{Title}

%\pagestyle{empty}
\usepackage{fancyhdr}
\usepackage{lastpage} 
  \pagestyle{fancy} 
   \let\origtitle\title 
  \renewcommand{\title}[1]{\lfoot{#1}\origtitle{#1}}

  \rfoot{\today}
  \rhead{[\ \scshape\oldstylenums{\thepage}\ / %
      \scshape\oldstylenums{\pageref{LastPage}}\ ]}
  \cfoot{}


\begin{document}

% the title page

\title{VDM++言語入門セミナー 演習問題\\
\small{($Revision: 0.2 $ -- \today)}}
\author{佐原 伸\\
法政大学\\
情報科学研究科
}
%\institute{\pgldk \and \chessnl}
\date{\mbox{}}
\maketitle

\begin{abstract}
\setlength{\baselineskip}{12pt plus .1pt}
%VDM++言語\cite{Kyushu2016PP}入門セミナーの演習問題である。
「対象を如何にモデル化するか?」のモデル作成例。
\end{abstract}
%\vspace{-1cm}

\tableofcontents

%\mainmatter
%\section {図書館0ビジネスロジック}
\include{Library0.vpp}
%\section {図書館0要求辞書}
\include{Library0RQ1.vdmpp}
%\section {図書館1ビジネスロジック}
\include{Library1.vpp}
%\section {図書館1要求辞書}
\include{LibraryRQ1.vdmpp}
%\section {図書館1回帰テストケース}
\include{MyTest.vpp}
\include{MyTestCase.vpp}

%\section {図書館2ビジネスロジック}
\include{Library2.vpp}
%\section {図書館2要求辞書}
\include{Author2.vdmpp}
\include{Book2.vpp}
\include{BookStacks.vdmpp}
\include{Field2.vdmpp}
\include{Lend2.vdmpp}
\include{Person2.vdmpp}
\include{Staff2.vdmpp}
\include{User2.vdmpp}
%\section {図書館2回帰テストケース}
\include{MyTest2.vpp}
\include{MyTestCase2.vpp}

%\section {ユーティリティ SSlib}
%\include{Character.vpp}
%% LaTeX 2e Document.
% 
% $Id: Sequence.tex,v 1.2 2006/01/10 10:45:26 vdmtools Exp $
% 

%%%%%%%%%%%%%%%%%%%%%%%%%%%%%%%%%%%%%%%%
% PDF compatibility code. 
\makeatletter
\newif\ifpdflatex@
\ifx\pdftexversion\@undefined
\pdflatex@false
%\message{Not using pdf}
\else
\pdflatex@true
%\message{Using pdf}
\fi

\newcommand{\latexorpdf}[2]{
  \ifpdflatex@ #2
  \else #1
  \fi
}

\newcommand{\pformat}{a4paper}

\makeatother
%%%%%%%%%%%%%%%%%%%%%%%%%%%%%%%%%%%%%%%%

\latexorpdf{
\documentclass[\pformat,12pt]{jarticle}
}{
\documentclass[\pformat,pdftex,12pt]{jarticle}
}


\usepackage[dvips]{color}
\usepackage{longtable}
\usepackage{alltt}
\usepackage{graphics}
\usepackage{vpp}
\usepackage{makeidx}
\makeindex

\definecolor{covered}{rgb}{0,0,0}      %black
%\definecolor{not-covered}{gray}{0.5}   %gray for previewing
%\definecolor{not-covered}{gray}{0.6}   %gray for printing
\definecolor{not-covered}{rgb}{1,0,0}  %red

\newcommand{\InstVarDef}[1]{{\bf #1}}
\newcommand{\TypeDef}[1]{{\bf #1}}
\newcommand{\TypeOcc}[1]{{\it #1}}
\newcommand{\FuncDef}[1]{{\bf #1}}
\newcommand{\FuncOcc}[1]{#1}
\newcommand{\MethodDef}[1]{{\bf #1}}
\newcommand{\MethodOcc}[1]{#1}
\newcommand{\ClassDef}[1]{{\sf #1}}
\newcommand{\ClassOcc}[1]{#1}
\newcommand{\ModDef}[1]{{\sf #1}}
\newcommand{\ModOcc}[1]{#1}


\title{Sequence}
\author{佐原伸}
\date{2005年3月22日}

\begin{document}
\maketitle

\section{Introduction}

Sequenceライブラリ。

% LaTeX 2e Document.
% 
% $Id: Sequence.tex,v 1.2 2006/01/10 10:45:26 vdmtools Exp $
% 

%%%%%%%%%%%%%%%%%%%%%%%%%%%%%%%%%%%%%%%%
% PDF compatibility code. 
\makeatletter
\newif\ifpdflatex@
\ifx\pdftexversion\@undefined
\pdflatex@false
%\message{Not using pdf}
\else
\pdflatex@true
%\message{Using pdf}
\fi

\newcommand{\latexorpdf}[2]{
  \ifpdflatex@ #2
  \else #1
  \fi
}

\newcommand{\pformat}{a4paper}

\makeatother
%%%%%%%%%%%%%%%%%%%%%%%%%%%%%%%%%%%%%%%%

\latexorpdf{
\documentclass[\pformat,12pt]{jarticle}
}{
\documentclass[\pformat,pdftex,12pt]{jarticle}
}


\usepackage[dvips]{color}
\usepackage{longtable}
\usepackage{alltt}
\usepackage{graphics}
\usepackage{vpp}
\usepackage{makeidx}
\makeindex

\definecolor{covered}{rgb}{0,0,0}      %black
%\definecolor{not-covered}{gray}{0.5}   %gray for previewing
%\definecolor{not-covered}{gray}{0.6}   %gray for printing
\definecolor{not-covered}{rgb}{1,0,0}  %red

\newcommand{\InstVarDef}[1]{{\bf #1}}
\newcommand{\TypeDef}[1]{{\bf #1}}
\newcommand{\TypeOcc}[1]{{\it #1}}
\newcommand{\FuncDef}[1]{{\bf #1}}
\newcommand{\FuncOcc}[1]{#1}
\newcommand{\MethodDef}[1]{{\bf #1}}
\newcommand{\MethodOcc}[1]{#1}
\newcommand{\ClassDef}[1]{{\sf #1}}
\newcommand{\ClassOcc}[1]{#1}
\newcommand{\ModDef}[1]{{\sf #1}}
\newcommand{\ModOcc}[1]{#1}


\title{Sequence}
\author{佐原伸}
\date{2005年3月22日}

\begin{document}
\maketitle

\section{Introduction}

Sequenceライブラリ。

% LaTeX 2e Document.
% 
% $Id: Sequence.tex,v 1.2 2006/01/10 10:45:26 vdmtools Exp $
% 

%%%%%%%%%%%%%%%%%%%%%%%%%%%%%%%%%%%%%%%%
% PDF compatibility code. 
\makeatletter
\newif\ifpdflatex@
\ifx\pdftexversion\@undefined
\pdflatex@false
%\message{Not using pdf}
\else
\pdflatex@true
%\message{Using pdf}
\fi

\newcommand{\latexorpdf}[2]{
  \ifpdflatex@ #2
  \else #1
  \fi
}

\newcommand{\pformat}{a4paper}

\makeatother
%%%%%%%%%%%%%%%%%%%%%%%%%%%%%%%%%%%%%%%%

\latexorpdf{
\documentclass[\pformat,12pt]{jarticle}
}{
\documentclass[\pformat,pdftex,12pt]{jarticle}
}


\usepackage[dvips]{color}
\usepackage{longtable}
\usepackage{alltt}
\usepackage{graphics}
\usepackage{vpp}
\usepackage{makeidx}
\makeindex

\definecolor{covered}{rgb}{0,0,0}      %black
%\definecolor{not-covered}{gray}{0.5}   %gray for previewing
%\definecolor{not-covered}{gray}{0.6}   %gray for printing
\definecolor{not-covered}{rgb}{1,0,0}  %red

\newcommand{\InstVarDef}[1]{{\bf #1}}
\newcommand{\TypeDef}[1]{{\bf #1}}
\newcommand{\TypeOcc}[1]{{\it #1}}
\newcommand{\FuncDef}[1]{{\bf #1}}
\newcommand{\FuncOcc}[1]{#1}
\newcommand{\MethodDef}[1]{{\bf #1}}
\newcommand{\MethodOcc}[1]{#1}
\newcommand{\ClassDef}[1]{{\sf #1}}
\newcommand{\ClassOcc}[1]{#1}
\newcommand{\ModDef}[1]{{\sf #1}}
\newcommand{\ModOcc}[1]{#1}


\title{Sequence}
\author{佐原伸}
\date{2005年3月22日}

\begin{document}
\maketitle

\section{Introduction}

Sequenceライブラリ。

\include{Sequence.vpp}


\newpage
\addcontentsline{toc}{section}{Index}
\printindex


\end{document}



\newpage
\addcontentsline{toc}{section}{Index}
\printindex


\end{document}



\newpage
\addcontentsline{toc}{section}{Index}
\printindex


\end{document}

%% LaTeX 2e Document.
% 
% $Id: String.tex,v 1.2 2006/01/10 10:45:26 vdmtools Exp $
% 

%%%%%%%%%%%%%%%%%%%%%%%%%%%%%%%%%%%%%%%%
% PDF compatibility code. 
\makeatletter
\newif\ifpdflatex@
\ifx\pdftexversion\@undefined
\pdflatex@false
%\message{Not using pdf}
\else
\pdflatex@true
%\message{Using pdf}
\fi

\newcommand{\latexorpdf}[2]{
  \ifpdflatex@ #2
  \else #1
  \fi
}

\newcommand{\pformat}{a4paper}

\makeatother
%%%%%%%%%%%%%%%%%%%%%%%%%%%%%%%%%%%%%%%%

\latexorpdf{
\documentclass[\pformat,12pt]{jsarticle}
}{
\documentclass[\pformat,pdftex,12pt]{jsarticle}
}


\usepackage[dvips]{color}
\usepackage{longtable}
\usepackage{alltt}
\usepackage{graphics}
\usepackage{vpp}
\usepackage{makeidx}
\makeindex

\definecolor{covered}{rgb}{0,0,0}      %black
%\definecolor{not-covered}{gray}{0.5}   %gray for previewing
%\definecolor{not-covered}{gray}{0.6}   %gray for printing
\definecolor{not-covered}{rgb}{1,0,0}  %red

\newcommand{\InstVarDef}[1]{{\bf #1}}
\newcommand{\TypeDef}[1]{{\bf #1}}
\newcommand{\TypeOcc}[1]{{\it #1}}
\newcommand{\FuncDef}[1]{{\bf #1}}
\newcommand{\FuncOcc}[1]{#1}
\newcommand{\MethodDef}[1]{{\bf #1}}
\newcommand{\MethodOcc}[1]{#1}
\newcommand{\ClassDef}[1]{{\sf #1}}
\newcommand{\ClassOcc}[1]{#1}
\newcommand{\ModDef}[1]{{\sf #1}}
\newcommand{\ModOcc}[1]{#1}


\title{VDM++ Sorting Algorithms}
\author{IFAD}
\date{August, 1997}

\begin{document}
\maketitle

\section{Introduction}

This document contains a sorting example. The class diagram can be
seen in Figure \ref{inh}.  The structure of the example is known as
the \textit{strategy} pattern. This pattern defines a family of
algorithms, encapsulates each one and make them interchangeable. The
\textit{strategy} pattern lets the algorithm vary independently from
clients that use it. The \texttt{SortMachine} class is the client that uses the
different sorting algorithms. The \texttt{Sorter} class is an abstract class
that defines a common interface to all supported algorithms.

% LaTeX 2e Document.
% 
% $Id: String.tex,v 1.2 2006/01/10 10:45:26 vdmtools Exp $
% 

%%%%%%%%%%%%%%%%%%%%%%%%%%%%%%%%%%%%%%%%
% PDF compatibility code. 
\makeatletter
\newif\ifpdflatex@
\ifx\pdftexversion\@undefined
\pdflatex@false
%\message{Not using pdf}
\else
\pdflatex@true
%\message{Using pdf}
\fi

\newcommand{\latexorpdf}[2]{
  \ifpdflatex@ #2
  \else #1
  \fi
}

\newcommand{\pformat}{a4paper}

\makeatother
%%%%%%%%%%%%%%%%%%%%%%%%%%%%%%%%%%%%%%%%

\latexorpdf{
\documentclass[\pformat,12pt]{jsarticle}
}{
\documentclass[\pformat,pdftex,12pt]{jsarticle}
}


\usepackage[dvips]{color}
\usepackage{longtable}
\usepackage{alltt}
\usepackage{graphics}
\usepackage{vpp}
\usepackage{makeidx}
\makeindex

\definecolor{covered}{rgb}{0,0,0}      %black
%\definecolor{not-covered}{gray}{0.5}   %gray for previewing
%\definecolor{not-covered}{gray}{0.6}   %gray for printing
\definecolor{not-covered}{rgb}{1,0,0}  %red

\newcommand{\InstVarDef}[1]{{\bf #1}}
\newcommand{\TypeDef}[1]{{\bf #1}}
\newcommand{\TypeOcc}[1]{{\it #1}}
\newcommand{\FuncDef}[1]{{\bf #1}}
\newcommand{\FuncOcc}[1]{#1}
\newcommand{\MethodDef}[1]{{\bf #1}}
\newcommand{\MethodOcc}[1]{#1}
\newcommand{\ClassDef}[1]{{\sf #1}}
\newcommand{\ClassOcc}[1]{#1}
\newcommand{\ModDef}[1]{{\sf #1}}
\newcommand{\ModOcc}[1]{#1}


\title{VDM++ Sorting Algorithms}
\author{IFAD}
\date{August, 1997}

\begin{document}
\maketitle

\section{Introduction}

This document contains a sorting example. The class diagram can be
seen in Figure \ref{inh}.  The structure of the example is known as
the \textit{strategy} pattern. This pattern defines a family of
algorithms, encapsulates each one and make them interchangeable. The
\textit{strategy} pattern lets the algorithm vary independently from
clients that use it. The \texttt{SortMachine} class is the client that uses the
different sorting algorithms. The \texttt{Sorter} class is an abstract class
that defines a common interface to all supported algorithms.

% LaTeX 2e Document.
% 
% $Id: String.tex,v 1.2 2006/01/10 10:45:26 vdmtools Exp $
% 

%%%%%%%%%%%%%%%%%%%%%%%%%%%%%%%%%%%%%%%%
% PDF compatibility code. 
\makeatletter
\newif\ifpdflatex@
\ifx\pdftexversion\@undefined
\pdflatex@false
%\message{Not using pdf}
\else
\pdflatex@true
%\message{Using pdf}
\fi

\newcommand{\latexorpdf}[2]{
  \ifpdflatex@ #2
  \else #1
  \fi
}

\newcommand{\pformat}{a4paper}

\makeatother
%%%%%%%%%%%%%%%%%%%%%%%%%%%%%%%%%%%%%%%%

\latexorpdf{
\documentclass[\pformat,12pt]{jsarticle}
}{
\documentclass[\pformat,pdftex,12pt]{jsarticle}
}


\usepackage[dvips]{color}
\usepackage{longtable}
\usepackage{alltt}
\usepackage{graphics}
\usepackage{vpp}
\usepackage{makeidx}
\makeindex

\definecolor{covered}{rgb}{0,0,0}      %black
%\definecolor{not-covered}{gray}{0.5}   %gray for previewing
%\definecolor{not-covered}{gray}{0.6}   %gray for printing
\definecolor{not-covered}{rgb}{1,0,0}  %red

\newcommand{\InstVarDef}[1]{{\bf #1}}
\newcommand{\TypeDef}[1]{{\bf #1}}
\newcommand{\TypeOcc}[1]{{\it #1}}
\newcommand{\FuncDef}[1]{{\bf #1}}
\newcommand{\FuncOcc}[1]{#1}
\newcommand{\MethodDef}[1]{{\bf #1}}
\newcommand{\MethodOcc}[1]{#1}
\newcommand{\ClassDef}[1]{{\sf #1}}
\newcommand{\ClassOcc}[1]{#1}
\newcommand{\ModDef}[1]{{\sf #1}}
\newcommand{\ModOcc}[1]{#1}


\title{VDM++ Sorting Algorithms}
\author{IFAD}
\date{August, 1997}

\begin{document}
\maketitle

\section{Introduction}

This document contains a sorting example. The class diagram can be
seen in Figure \ref{inh}.  The structure of the example is known as
the \textit{strategy} pattern. This pattern defines a family of
algorithms, encapsulates each one and make them interchangeable. The
\textit{strategy} pattern lets the algorithm vary independently from
clients that use it. The \texttt{SortMachine} class is the client that uses the
different sorting algorithms. The \texttt{Sorter} class is an abstract class
that defines a common interface to all supported algorithms.

\include{String.vpp}

\newpage
\addcontentsline{toc}{section}{Index}
\printindex


\end{document}


\newpage
\addcontentsline{toc}{section}{Index}
\printindex


\end{document}


\newpage
\addcontentsline{toc}{section}{Index}
\printindex


\end{document}


\newpage
%\layout

%\begin{thebibliography}{9}
\section{参考文献、索引}
VDM++\cite{Kyushu2016PP}は、
1970年代中頃にIBMウィーン研究所で開発されたVDM-SL\cite{Kyushu2016SL}を拡張し、
さらにオブジェクト指向拡張した形式仕様記述言語である。


VDM++の教科書としては\cite{Sakoh2010}がある。

VDM++を開発現場で実践的に使う場合の解説が\cite{Sahara2008}にある。

\bibliographystyle{jplain}
%\bibliography{/Users/sahara/svnw/sahara}
\bibliography{/Users/sahara/Dropbox/bib/saharaUTF8}
%\bibliography{/Users/ssahara/svnwork/sahara}

%\end{thebibliography}

%\newpage
%\addcontentsline{toc}{section}{Index}
\printindex

\end{document}
