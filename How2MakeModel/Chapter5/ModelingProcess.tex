対象を如何にモデル化するかということは、
こうやれば誰でも確実に良いモデルができるという一般的手順があるわけではない。
しかし、数学やソフトウェア工学の歴史の中で培われた一般的原則と手順はある。
本章では、このような一般的原則と手順を紹介する。
何らかの仕様記述言語を前提としないと説明が難しいので、VDM++言語を前提とする。

これらの原則と手順は、
構造化分析・設計手法(SA/SD)やオブジェクト指向分析・設計手法(OOA/OOD)におけるモデル化手法と、
より品質を高める形式手法の技術を除けば、さほど違いがあるわけではない。
	\index{こうぞうかぶんせきせっけいしゅほう@構造化分析・設計手法(SA/SD)}
	\index{おぶじぇくとしこうぶんせきせっけいしゅほう@オブジェクト指向分析・設計手法(OOA/OOD)}

\section {モデル化の一般的手順}
\index{もでるかのいっぱんてきてじゅん@モデル化の一般手順}

\begin{enumerate}
\item モデル化の範囲を決める。
	\index{もでるかのはんいをきめる@モデル化の範囲を決める}

	例えば、「エラー処理の詳細を除くユースケースレベルの要求仕様を書く」とか、
	「アプリケーションレベルのエラー処理の詳細も含めて、外部仕様相当の設計仕様を記述する」といった方針を決める。
	過去に成功した形式手法使用プロジェクトの調査結果\cite{SEC2012FMREPORT}では、この範囲決めが重要で、
	かつ、この範囲内をすべて形式仕様で記述することが形式手法採用プロジェクト成功の鍵であることが分かっている。

\item 仕様を分割統治する。
	\index{しようをぶんかつとうちする@仕様を分割統治する}

	階層化とモジュール化(\ref{SpecFramework}節と\ref{sec:specFM}節で紹介する仕様記述フレームワークを使う)を行い、
	ある階層やモジュールの変更余波が、他の階層やモジュールに及ばないようにモデルを構築する。

	これによって、モデルの保守性や再利用性が増し、可読性も増す。

\item 主に名詞から型を定義する。
	\index{めいしからかたをていぎする@名詞から型を定義する}

	属性の参照に留まらない機能を持つ型は、クラスとする。

\item システムが状態を持つ場合、その状態を定義する。
	\index{じょうたいをていぎする@状態を定義する}

	VDM++の場合、状態はインスタンス変数として持つ。

\item 主に、述語(動詞)から、関数・操作のインタフェース(シグネチャという)を記述する。
	\index{じゅつごからかんすうそうさのいんたふぇーすをていぎする@述語から、関数・操作のインタフェースを記述する}

\item 関数・操作の事後条件・事前条件を記述する(陰仕様
		\footnote{関数・操作のインタフェースと、事前条件及び事後条件しか記述していない仕様を陰仕様と呼ぶ。
		陰仕様は、関数・操作の本体は記述していないが、「仕様」を記述していることになる。}
	作成)。
	\index{いんしようをさくせいする@陰仕様を作成する}

	事後条件は、関数ないし操作が終わった状態を真(true)を返す論理式で表す。
	したがって、関数ないし操作の「仕様」と言える。

	事前条件は、引数の制約条件を論理式で記述し、仕様の責任を明確化するためのものである。
	すなわち、事前条件が偽となる引数に対して責任を負わないことを宣言している。
	操作の場合は、インスタンス変数の制約条件も記述することができる。

\item 静的検証を行う。
	\index{せいてきけんしょうをおこなう@静的検証を行う}

	構文・型・証明課題チェックをツールを使用して行う。
	この段階で、多くの単純ミスが検出され、仕様作成者は「仕様の意味」に集中することができる。
	

\item 関数・操作の本体を記述する(陽仕様
		\footnote{関数・操作の本体が記述されていて実行可能な仕様を陽仕様と呼ぶ。}
	にする)。
	\index{ようしようをさくせいする@陽仕様を作成する}

	陰仕様は、静的検証以外の検証ができないため、事後条件を満たすか否かの動的検証を行うために、
	関数・操作の本体を記述する。

	事後条件を満たす本体の記述は幾通りもあり得るし、本体の記述は一つの実装を表しているので、
	本体は、仕様そのものではなく、事後条件で記述された「仕様」が正しいことを検証するためのものである、と言える。
	

\item モデルの正当性と妥当性を、動的に検証・確認する。
	\index{もでるのせいとうせいとだとうせいをどうてきにけんしょうかくにんする@モデルの正当性と妥当性を動的に検証・確認する}

	ツールのインタープリタ、デバッガを使用して、モデルの正当性(verification)を検証し妥当性(validation)を確認する。

\item 要求仕様をレビューし、漏れがないか再検討する。
	\index{ようきゅうしようをれびゅーする@要求仕様をレビュー}

	形式手法を使っていても、仕様作成者やユーザーによるレビューは必要である。
	意味的な仕様の整合性は、人間しか気が付かない事が多いからである。
\end{enumerate}