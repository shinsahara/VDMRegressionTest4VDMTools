	ここまで、なぜ形式手法か?、VDMの概要・成果・導入方法、如何にして対象をモデル化するか?を説明してきたが、
	本章では、ここまで説明した内容を、
	形式手法とVDMの有用性、構造化日本語仕様としてのVDM++という観点からまとめる。

\section{形式手法とVDMの有用性}
	\label{FMEfficency}
	\index{けいしきしゅほうとVDMのゆうようせい@形式手法とVDMの有用性}

	\begin{itemize}
	\item 理論的にも経験的にも形式手法が有効 \\
		証明やモデル検査は、長期的にはやるべきだが、開発現場で導入するのはすぐには難しい。
	\item 証明やモデル検査と比べ、VDM導入はさほど難しくない
		\begin{itemize}
		\item 成功プロジェクト
			\footnote{厳密な仕様記述における形式手法成功事例調査報告書
				(\url{http://sec.ipa.go.jp/reports/20130125.html})
				のTradeOneとFeliCaファームウェアのプロジェクト参照のこと。}
			は、3ヶ月程度の教育とコンサルティングで導入している
		\item 形式手法だけでなく、既存の役立つソフトウェア工学ツールと協調して、より効果が出る
		\end{itemize} 
	\item モデル化は、以下が重要である
		\begin{itemize}
		\item モデル化の範囲を決め、仕様を階層と分割によって、分割統治する
		\item 名詞から型、述語から関数または操作を導き出す
		\item 陰仕様を作成してから、静的検証を行う \\
			この段階で、間違いに気付くことが非常に多い。
		\item 陽仕様を作成してから、動的検証を行う \\
			回帰テストと組合せることで、仕様の変更が容易になる。
		\end{itemize} 
	\item VDM++ソースは、識別子の日本語表現を適切に行うと、構造化日本語仕様として使うことができる
	\end{itemize} 

\section{構造化日本語仕様としてのVDM}
	\label{VDMAsStructuredJapanese}
	\index{こぞうかにほんごとしてのVDM@構造化日本語仕様としてのVDM}

	\ref{ExpressReservationSystem}節\ref{Reserve}段落の操作は、以下のようであり、若干のVDM++知識があれば、
	構造化された日本語仕様として読むことができる。

\begin{verbatim}
public 予約する :  契約 * ID * クレジットカード * 特急券予約`予約内容 ==> 特急券予約
予約する(a契約, anID, aクレジットカード, a予約内容) == (
   def w特急券予約 = new 特急券予約(anID, aクレジットカード, a予約内容) in (
   if 予約がある契約である(a契約, s予約表) then
      s予約表 := 予約表を更新する(s予約表, a契約, w特急券予約)
   else
      s予約表 := 予約表に追加する(s予約表, a契約, w特急券予約);
   return w特急券予約
   )
)
post
   if 予約がある契約である(a契約, s予約表~) then
      予約表が更新されている(s予約表~, a契約, RESULT, s予約表)
   else
      予約表に追加されている(s予約表~, a契約, RESULT, s予約表);
\end{verbatim}

このVDM++仕様から、
事後条件として、
「予約がある契約である」ならば、「予約表が更新されている」状態になり、
そうでなければ、「予約表に追加されている」状態になる
事が分かる。
「予約表が更新されている」ことと「予約表に追加されている」ことの違いは、
それぞれの関数の内容を確認すれば理解できる。

VDM++の知識が少しあれば、
パラメータとして「予約表が更新されている」と「予約表に追加されている」ものに、
インスタンス変数であるs予約表の旧値(old value)と現在の値が渡されていて、
恐らくそれぞれの関数の中で両者の比較が行われていることも分かるし、
返り値を表す予約後のRESULTがパラメータとして渡されることから、
返り値と何かを比較しているだろうことも分かる。

表\ref{PseudocodeVsVDM}に、開発現場でよく使われる、
擬似コードを交えた日本語仕様と適切な日本語を使って構造化されたVDM++仕様の比較を示す。

\begin{table}[h]
	\caption[日本語の仕様中の擬似コードと、VDMの比較]{日本語の仕様中の擬似コードと、VDMの比較}
	\index{にほんごしようちゅうのぎじこーどとVDMのひかく@日本語の仕様中の擬似コードと、VDMの比較}
	\label{PseudocodeVsVDM}
	\begin{center}
		\setlength{\tabcolsep}{3pt}
		\begin{tabular}{|p{5zw}|p{11zw}|p{8zw}|p{9zw}|p{11zw}|p{6zw}|} \hline
			 & 構文を考える時間 & 構文チェック & 関連チェック & 実行テスト & 証明課題生成  \\ \hline\hline
			擬似コードによる仕様  & かなりの時間が必要で、記法も統一できない & レビューのみ & レビューのみ & 
				具体的データを想定したコードインスペクションに相当するチェックのみ(通常行われない) & 不可能 \\ \hline
			VDM仕様  & 言語マニュアルを参照すれば良いだけ & ツールでチェック & ツールの型チェック & 
				ツールで実行 & ツールで生成 \\ \hline
		\end{tabular}
	\end{center}
\end{table}

	表\ref{PseudocodeVsVDM}を見れば、擬似コードを混じえた日本語仕様は「使えない」ことが分かる。

	適切な日本語仕様を使って構造化されたVDM++仕様、
	すなわち構造化日本語仕様としてのVDM仕様は、以下の特徴を持つ。

	\begin{itemize}
	\item 構文を考える時間が不要である \\
		逆に、擬似コードを使う日本語仕様では、
		仕様の意味を考えねばならないのに、構文を考えていることが非常に多い。
	\item 構文・型チェック、証明課題レビューで静的に検証できる \\
		非常に多くの単純ミスを簡単に発見できる。
	\item 証明課題で生成される条件式から、見落としていた不変条件や事前条件が見つかる
	\item 組合せテストにより、動的に正当性検証ができる
	\item 回帰テストにより、動的に妥当性確認ができる
	\item VDMソース自体が、要求辞書ともなる
	\item 日本語仕様より、記述と検証の工数が少ない \\
		特に、仕様修正に強い
	\item 擬似コード形式の日本語仕様に近い形で、かなりの部分を記述できる
	\end{itemize} 

\section{形式手法導入のコツ}
	\label{How2UseFM}
	\index{けいしきしゅほうどうにゅうのこつ@形式手法導入のコツ}

	ここまで説明してきたことから、形式手法導入のコツは極めて簡単であることが分かる。

	すなわち、日本で最大規模の形式仕様記述を使ってプロジェクトを成功に導いた、
	フェリカネットワークスの栗田太郎氏が述べているように、「やれば良いだけです」。
	本書が、そのお役に立てることを願っている。